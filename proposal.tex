\documentclass[11pt, oneside]{amsart} % use "amsart" | "article" for format
\usepackage{geometry}
\geometry{letterpaper}
\usepackage{graphicx} % Use pdf, png, jpg, or eps§ with pdflatex
\usepackage{amssymb}

\title{Title}
% title: - TBD
%{Scene Classification with Deep Convolutional Neural Networks} or {Deep Convolutional Neural Networks for Recognizing Natural Scene Categories}
\author{The Author}
\date{} % Activate to display a given date or no date

\begin{document}
\maketitle

% overview: describe the problem and main idea
\section{Problem}
	In this project, we are planning to recognize the semantic category of an image. Developing robust and geometrically invariant object representation remain a challenging problems in machine intelligence. 

% related work: briefly describe related papers
\section{Previous Work}
Todo

% technical approach: describe the feature representation(s) and algorithm(s) you will employ
\section{Approach}
\par
A. Krizhevsky et al. trained a large, deep convolutional neural network to classify high-resolution images in ImageNet contest and obtained impressive results for highly challenging object recognition tasks using supervised learning. The features using this technique can also be used for scene categories. 

\par
Selective search can be used for generating possible object locations for object recognition or scene classification. This helps us reduce the processing time significantly without lose overall performance. 

\par	
Constructing three-level pyramid allows us to precisely matching of two collections of features in a high-dimensional appearance space despite of the spatial information.

\par
The scene classification will be done with a linear support vector machine (SVM).

% experiments: Describe the experiments to evaluate your approach
\section{Experiments}

% others: describe software, libraries, language that you will use
% and how you plan to share the work with your partner.
\section{Matirials}
	\subsection{Dataset}
	MIT indoor 67 dataset
	\subsection{Tools}
	Caffe, Selective Search Software, LibSVM
 
% references:
 
% Beyond Bags of Features: Spatial Pyramid Matching for Recognizing Natural Scene Categories. S. Lazebnik, C. Schmid, and J. Ponce. CVPR 2006.
 
% Object Bank: A High-Level Image Representation for Scene Classi?cation & Semantic Feature Sparsi?cation. L-J. Li, H. Su, E. Xing, and L. Fei-Fei. NIPS 2010.
 
% ImageNet Classification with Deep Convolutional Neural Networks. A. Krizhevsky, I. Sutskever, and G. Hinton. NIPS 2012.
 
% Selective Search for Object Recognition
%J. R. R. Uijlings, K. E. A. van de Sande, T. Gevers, A. W. M. Smelters
%In International Journal of Computer Vision 2013. 
 
% LibSVM: http://www.csie.ntu.edu.tw/~cjlin/libsvm/
 
% http://caffe.berkeleyvision.org/
 
% MIT dataset: A. Quattoni, and A.Torralba. Recognizing Indoor Scenes. IEEE Conference on Computer Vision and Pattern Recognition (CVPR), 2009.

\end{document}  