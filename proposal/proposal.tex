\documentclass[letterpaper,twocolumn,11pt]{article}

\usepackage[latin1]{inputenc}
\usepackage[english]{babel}
\usepackage{scs}
\usepackage{graphicx}
\usepackage{cite}
\usepackage{amssymb,amsmath}
\usepackage{url}
\usepackage{multicol}

% title: - TBD
\title{Scene Classification with Deep Convolutional Neural Networks}
%{Scene Classification with Deep Convolutional Neural Networks}
%{Deep Convolutional Neural Networks for Recognizing Natural Scene Categories}
\author{Yangzihao Wang, Yuduo Wu}
\date{} % Activate to display a given date or no date

\begin{document}
\maketitle

% overview: describe the problem and main idea
\section{Problem}
High-level image recognition is one of the most challenging domains in the field of computer vision. Several works have been done trying to close this semantic gap. Developing robust and geometrically invariant feature representation is critical to the performance of a successful learning system. Recently, neural networks have grown to be one of the best-performing methods in visual recognition field. In this project, we plan to combine the output features of a trained Convolutional Neural Network (CNN) and spatial pyramid matching scheme to create a novel feature representation and use it to recognize the semantic categories of images.


% related work: briefly describe related papers
\section{Previous Work}
Talk about three works: Object Bank, Spatial Pyramid, ImageNet CNN.

% technical approach: describe the feature representation(s) and algorithm(s)
\section{Approach}
\par
	A. Krizhevsky et al.\cite{CNN} trained a large, deep convolutional neural network used for classifying high-resolution images in ImageNet contest
	and obtained impressive results for highly challenging object recognition
	tasks using	supervised learning. The features using this technique can also
	be used	for scene categories.

\par
	Selective search\cite{SS} can be used for generating possible object
	locations for object recognition or scene classification. This helps us
	reduce the processing time significantly without lose overall performance.

\par
	Constructing three-level pyramid allows us to precisely matching of two
	collections of features in a high-dimensional appearance space despite
	of the spatial information.\cite{SPM}

\par
	The scene classification will be done with a linear support vector machine
	(SVM).\cite{SVM}

% experiments: Describe the experiments to evaluate your approach
\section{Experiments}


% others: describe software, libraries, language that you will use
% and how you plan to share the work with your partner.
\section{Matirials}
	\subsection{Dataset}
	MIT indoor 67 dataset\cite{DATA}

	\subsection{Tools}
	Caffe\cite{CAFFE}, Selective Search Software\cite{SS}, LibSVM\cite{SVM}

% references:
\begin{thebibliography}{10}

\bibitem{CNN} A. Krizhevsky, I. Sutskever, and G. Hinton.,
"ImageNet Classification with Deep Convolutional Neural Networks",
NIPS 2012.

\bibitem{SS} J. R. R. Uijlings, K. E. A. van de Sande,
T. Gevers, A. W. M. Smelters
"Selective Search for Object Recognition",
\emph{In International Journal of Computer Vision}, 2013.

\bibitem{SPM} S. Lazebnik, C. Schmid, and J. Ponce.,
"Beyond Bags of Features:
Spatial Pyramid Matching for Recognizing Natural Scene Categories",
CVPR 2006.

\bibitem{OB} L-J. Li, H. Su, E. Xing, and L. Fei-Fei.,
"Object Bank: A High-Level Image Representation for Scene Classification
\& Semantic Feature Sparsification",
NIPS 2010.

\bibitem{SVM} LibSVM: http://www.csie.ntu.edu.tw/~cjlin/libsvm/

\bibitem{CAFFE} Caffe: http://caffe.berkeleyvision.org/

\bibitem{DATA} A. Quattoni, and A.Torralba. Recognizing Indoor Scenes.
\emph{IEEE Conference on Computer Vision and Pattern Recognition (CVPR)},
2009.

\end{thebibliography}
\end{document}