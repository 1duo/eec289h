\documentclass[letterpaper,twocolumn,10pt]{article}

\usepackage[latin1]{inputenc}
\usepackage[english]{babel}
\usepackage{scs}
\usepackage{graphicx}
\usepackage{cite}
\usepackage{amssymb,amsmath}
\usepackage{url}
\usepackage{multicol}

% title: - TBD
\title{Scene Classification with Deep Convolutional Neural Networks}
%{Scene Classification with Deep Convolutional Neural Networks}
%{Deep Convolutional Neural Networks for Recognizing Natural Scene Categories}
\author{Yangzihao Wang, Yuduo Wu}
\date{} % Activate to display a given date or no date

\begin{document}
\maketitle

% overview: describe the problem and main idea
\section{Problem}
High-level image recognition is one of the most challenging domains in the
field of computer vision. Several works have been done trying to close this
semantic gap. Developing robust and geometrically invariant feature
representation is critical to the performance of a successful learning system.
Recently, neural networks have grown to be one of the best-performing methods
in visual recognition field. In this project, we plan to combine the output
features of a trained Convolutional Neural Network (CNN) and spatial pyramid
matching scheme to create a novel feature representation and use it to
recognize the semantic categories of images.

% related work: briefly describe related papers
\section{Related Works}
Talk about three works: Object Bank, Spatial Pyramid, ImageNet CNN.  Object
bank\cite{OB} is among the first to propose a high-level image representation.
It uses a large number of pre-trained generic object detectors to create
response map, which limited the performance of their system to the performance
of object detectors they choose. Spatial pyramid representation\cite{SPM} is
a popular method used for scene categorization tasks. It is a simple and
computationally efficient extension of an orderless bag-of-features image
representation. However, without a proper high-level feature representation,
such schemes often fail to offer sufficient semantic information of a scene.
Convolutional neural networks (CNN) with flexible capacity makes training from
large-scale dataset possible. A. Krizhevsky et al.\cite{CNN} trained one of the
largest CNN on the subsets of ImageNet and achieved best results in 2012. While
their CNN system focuses on object detection, the features generated can be
used for other applications such as scene classification. Several algorithms
try to address the problem of generating possible object locations in an image.
Selective search method combines the strength of both an exhaustive search and
segmentation and results in a small set of data-driven, class-independent, high
quality locations.

% technical approach: describe the feature representation(s) and algorithm(s)
\section{Approach}
\par
    For feature representation, we plan to use pre-trained ImageNet CNN and
    extract the last couple of hidden-layers, which contain either 4096 or
    1000 neurons. The 4096 or 1000 vector would serve as our feature
    descriptor.

\par
    For multi-scale candidate windows generating, we plan to use selective
    search method, the number of candidate windows chosen for each image can be
    a tunable parameter.

\par
    After we obtain the candidate windows per image and get the feature
    descriptor vector for each window. We build a response map using
    a multi-layer spatial pyramid as in Object Bank to combine each feature's
    information and creating a final high-level image representation.

\par
    As for the classification, the first thing we planning to attempt is
    a linear support vector machine (SVM). There are some useful open source
    SVM libraries that are publicly available.  If possible, we also want to
    explore using a multi-class one-vs-all SVM as our learning model.

% experiments: Describe the experiments to evaluate your approach
\section{Experiments}
We intend to test the correctness of each component of the system first.
For tuning the system, we intend to do the following experiments:
\begin{itemize}
\item Use different layer's output of the CNN as our feature descriptor.
\item Test the number of candidate windows from each image to find the optimal one.
\item Test different spatial pyramid scheme (different layers, different ways to divide windows, and different pooling functions).
\item Test different SVMs.
\end{itemize}

% others: describe software, libraries, language that you will use
% and how you plan to share the work with your partner.
\section{Materials}
	\subsection{Dataset}
	\begin{itemize}
	\item{IndoorCVPR\_09} is a challenging dataset which contains 67
	indoor categories and a total of 15,620 images. To address these indoor
	scenes recognition, we need a model	that can exploit local and global
	discriminative information.\cite{DATA}
	\end{itemize}
	\subsection{Tools}
	\begin{itemize}
	\item{Caffe} is a high-performance deep learning framework with GPU
	acceleration. \cite{CAFFE}

	\item{Selective Search Software} code is publicly available to use
	for capture possible locations for object recognition. \cite{SS}

	\item{LibSVM} is a library for SVM classification. \cite{SVM}
	\end{itemize}

% references:
\begin{thebibliography}{10}

\bibitem{CNN} A. Krizhevsky, I. Sutskever, and G. Hinton.,
"ImageNet Classification with Deep Convolutional Neural Networks",
NIPS 2012.

\bibitem{SS} J. R. R. Uijlings, K. E. A. van de Sande,
T. Gevers, A. W. M. Smelters
"Selective Search for Object Recognition",
\emph{In International Journal of Computer Vision}, 2013.

\bibitem{SPM} S. Lazebnik, C. Schmid, and J. Ponce.,
"Beyond Bags of Features:
Spatial Pyramid Matching for Recognizing Natural Scene Categories",
CVPR 2006.

\bibitem{OB} L-J. Li, H. Su, E. Xing, and L. Fei-Fei.,
"Object Bank: A High-Level Image Representation for Scene Classification
\& Semantic Feature Sparsification",
NIPS 2010.

\bibitem{DATA} A. Quattoni, and A.Torralba. Recognizing Indoor Scenes.
\emph{IEEE Conference on Computer Vision and Pattern Recognition (CVPR)},
2009.

\bibitem{SVM} LibSVM: http://www.csie.ntu.edu.tw/~cjlin/libsvm/

\bibitem{CAFFE} Caffe: http://caffe.berkeleyvision.org/


\end{thebibliography}
\end{document}
