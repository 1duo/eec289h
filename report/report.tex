\documentclass[10pt,twocolumn,letterpaper]{article}

\usepackage{cvpr}
\usepackage{times}
\usepackage{epsfig}
\usepackage{graphicx}
\usepackage{amsmath}
\usepackage{amssymb}

% Include other packages here, before hyperref.

% If you comment hyperref and then uncomment it, you should delete
% egpaper.aux before re-running latex.  (Or just hit 'q' on the first latex
% run, let it finish, and you should be clear).
\usepackage[pagebackref=true,breaklinks=true,letterpaper=true,colorlinks,bookmarks=false]{hyperref}

\cvprfinalcopy % *** Uncomment this line for the final submission

\def\cvprPaperID{42} % *** Enter the CVPR Paper ID here
\def\httilde{\mbox{\tt\raisebox{-.5ex}{\symbol{126}}}}

% Pages are numbered in submission mode, and unnumbered in camera-ready
\ifcvprfinal\pagestyle{empty}\fi
\begin{document}

%%%%%%%%% TITLE
\title{Scene Classification with Deep Convolutional Neural Networks}

\author{Yangzihao Wang\\
University of California, Davis\\
{\tt\small yzhwang@ucdavis.edu}
% For a paper whose authors are all at the same institution,
% omit the following lines up until the closing ``}''.
% Additional authors and addresses can be added with ``\and'',
% just like the second author.
% To save space, use either the email address or home page, not both
\and
Yuduo Wu\\
University of California, Davis\\
{\tt\small yudwu@ucdavis.edu}
}

\maketitle
%\thispagestyle{empty}

%%%%%%%%% ABSTRACT
\begin{abstract}
summarize the problem, main idea, and results;
\end{abstract}

%%%%%%%%% BODY TEXT
\section{Introduction}

%-------------------------------------------------------------------------
\subsection{Related Work}

provide a detailed description of related papers (not necessarily limited to
those in the schedule).  If you're proposing a new idea or extending an existing
approach, compare and contrast it with existing work.  If you're analyzing one
or two related techniques, describe how they relate to other relevant work;
~\cite{Li:2010:OBA, Lazebnik:2006:BBF, Zhou:2014:LDF, Krizhevsky:2012:ICD, Quattoni:2009:RIS, Uijlings:2013:SSO}

\subsection{Technical Approach}

Describe in detail the feature representation(s) and algorithm(s) you employed.
The description should be self-contained (i.e., the reader should not have to
rely on outside sources for your points to be clear), and should provide enough
detail so that the reader could re-implement the approach.  Clearly state the
method's input and output, and any assumptions or design choices;

\subsection{Experiments}

Describe the experiments you conducted to evaluate the approach.  For each
experiment, describe what you did, what was the main purpose of the experiment,
and what you learned from the results. Provide figures, tables, and qualitative
examples, as appropriate.

%-------------------------------------------------------------------------
\subsection{Conclusions}

briefly summarize the main idea and results, and possible future work.

{\small
\bibliographystyle{ieee}
\bibliography{report}
}

\end{document}
