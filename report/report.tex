\documentclass[10pt,twocolumn,letterpaper]{article}

\usepackage{cvpr}
\usepackage{times}
\usepackage{epsfig}
\usepackage{graphicx}
\usepackage{amsmath}
\usepackage{amssymb}

% Include other packages here, before hyperref.

% If you comment hyperref and then uncomment it, you should delete
% egpaper.aux before re-running latex.  (Or just hit 'q' on the first latex
% run, let it finish, and you should be clear).
\usepackage[pagebackref=true,breaklinks=true,letterpaper=true,colorlinks,bookmarks=false]{hyperref}

\cvprfinalcopy % *** Uncomment this line for the final submission

\def\cvprPaperID{42} % *** Enter the CVPR Paper ID here
\def\httilde{\mbox{\tt\raisebox{-.5ex}{\symbol{126}}}}

% Pages are numbered in submission mode, and unnumbered in camera-ready
\ifcvprfinal\pagestyle{empty}\fi
\begin{document}

%%%%%%%%% TITLE
\title{Scene Classification with Deep Convolutional Neural Networks}

\author{Yangzihao Wang\\
University of California, Davis\\
{\tt\small yzhwang@ucdavis.edu}
% For a paper whose authors are all at the same institution,
% omit the following lines up until the closing ``}''.
% Additional authors and addresses can be added with ``\and'',
% just like the second author.
% To save space, use either the email address or home page, not both
\and
Yuduo Wu\\
University of California, Davis\\
{\tt\small yudwu@ucdavis.edu}
}

\maketitle
%\thispagestyle{empty}

%%%%%%%%% ABSTRACT
\begin{abstract}
%summarize the problem, main idea, and results;
The use of massive datasets like ImageNet and the revival of Convolutional
Neural Networks (CNNs) for learning deep features has significantly improved
the performance of object recognition. However, performance at scene
classification has not achieved the same level of success since there is still
semantic gap between the deep features and the high-level context.  In this
project we proposed a novel scene classification method which combines CNN and
Spatial Pyramid to generate high-level context-aware features for one-vs-all
linear SVMs. Our method achieves the state-of-the-art result: 68.04\% accuracy
rate on MIT indoor67 dataset using only the deep features trained from
ImageNet. 
 
\end{abstract}

%%%%%%%%% BODY TEXT
\section{Related Work}
\label{sec:related}
%provide a detailed description of related papers (not necessarily limited to
%those in the schedule).  If you're proposing a new idea or extending an existing
%approach, compare and contrast it with existing work.  If you're analyzing one
%or two related techniques, describe how they relate to other relevant work;

Scene classification means to provide information about the semantic category
or the function of a given image. Among different kidn of scene classification
tasks, the indoor scene classification is considered to be one of the most
difficult since the lack of discriminative features and contexts at the high
level~\cite{Quattoni:2009:RIS}. Spatial pyramid
representation\cite{Lazebnik:2006:BBF} is a popular method used for scene
classification tasks. It is a simple and computationally efficient extension of
an orderless bag-of-features image representation. However, without a proper
high-level feature representation, such schemes often fail to offer sufficient
semantic information of a scene. Object bank\cite{Li:2010:OBA} is among the
first to propose a high-level image representation for scene classification. It
uses a large number of pre-trained generic object detectors to create response
maps for high level visual recognition tasks. The combination of off-the-shelf
object detectors and a simple linear prediction model with a sparse-coding
scheme achieves superior predictive power over similar linear prediction models
trained on conventional representations. However, this method also limits the
performance of their system to the performance of the object detectors they
choose. Recently, Convolutional Neural Networks (CNNs) with flexible capacity
makes training from large-scale dataset such as ImageNet~\cite{Deng:2009:IAL}
possible. In the work of A. Krizhevsky et al.\cite{Krizhevsky:2012:ICD}, they
trained one of the largest CNNs on the subsets of ImageNet and achieved better
results than any other state-of-the-art methods in 2012. While their CNN system
focuses on object detection, the features generated can be used for other
applications such as scene classification. Two types of improvements has been
done on top of their CNN works. The first type of improvement tries to address
the problem of generating possible object locations in an image. Selective
search method~\cite{Uijlings:2013:SSO} combines the strength of both an
exhaustive search and segmentation and results in a small set of data-driven,
class-independent, high quality locations. Girshick et al. propose the Regions
with CNN features (R-CNN) method~\cite{Girshick:2013:RFH} as a more effective
feature generation method. Alternatively, Zhou et al. try to increase the
performance of scene classification using CNN by creating a new scene-centric
database~\cite{Zhou:2014:LDF}.


\subsection{Technical Approach}

Describe in detail the feature representation(s) and algorithm(s) you employed.
The description should be self-contained (i.e., the reader should not have to
rely on outside sources for your points to be clear), and should provide enough
detail so that the reader could re-implement the approach.  Clearly state the
method's input and output, and any assumptions or design choices;

\subsection{Experiments}

Describe the experiments you conducted to evaluate the approach.  For each
experiment, describe what you did, what was the main purpose of the experiment,
and what you learned from the results. Provide figures, tables, and qualitative
examples, as appropriate.

%-------------------------------------------------------------------------
\subsection{Conclusions}

briefly summarize the main idea and results, and possible future work.

{\small
\bibliographystyle{ieee}
\bibliography{report}
}

\end{document}
