%Describe in detail the feature representation(s) and algorithm(s) you employed.
%The description should be self-contained (i.e., the reader should not have to
%rely on outside sources for your points to be clear), and should provide enough
%detail so that the reader could re-implement the approach.  Clearly state the
%method's input and output, and any assumptions or design choices;

\subsection{Selective Search}
Selective search is widely used for generating possible object locations for
use in object recognition\cite{Uijlings:2013:SSO}. Same strategies can be
adopted on indoor scene classification. For the indoor scenes that can be well
characterized by objects they contain, the selective search can exploit local
discriminative information with greatly reduced number of locations compared
to an exhaustive search.

\subsection{Feature Extraction}

\subsubsection{Max Pooling}

\subsection{Spatial Pyramid Matching}
For each image, a three-level spatial pyramid representation is used, resulting
$number images * number windows * (1 + 4 + 16)$ length feature vectors.