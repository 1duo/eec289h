Describe the experiments you conducted to evaluate the approach.  For each
%experiment, describe what you did, what was the main purpose of the experiment,
%and what you learned from the results. Provide figures, tables, and qualitative
%examples, as appropriate.

In this section, we evaluate our method on the MIT-indoor67 dataset. Suggested training and
testing list of images are used to do the training (80 images per class)
and validation (20 images per class). There are at least 100 images per
scene and all in jpeg format.

Multi-class classification is done with a support vector machine (SVM) trained
using one-versus-all rule, that is, each classifier is learned to separate each
class from the rest of classes. Test image is assigned the label of the
classifier with the highest response.Scene classification performance is
evaluated by average multi-class classification accuracy over all scene classes.

For comparison purposes, we implement with the same procedure but only use
the extracted layer 7 4096-dimensional feature vectors from Caffe. After we get
one feature vectors for each entire image, instead of perform spatial pyramid
and L2 normalization, we simply add labels and send them into the multi-class
SVMs. Validation image feature vectors are also generated in the same way.

% Table 1
\begin{table*}[ht]
	\caption{Comparison results on MIT-indoor67}
	\centering
	\begin{tabular}{l c c}
	\hline \hline
	Models                & Average Precision \\ \hline
	$l2$ Norm + Selective Search + Spatial Pyramid & {\bf{68.2953\%}} \\
	Selective Search + Spatial Pyramid & 68.0469\% \\
        Entire Image CNN Features & 59.9507\% \\
	\hline
	\end{tabular}
	\label{tab:overall}
\end{table*}

Our
method achieves a mean average precision (mAP) of 68.2953\% on dataset
MIT-indoor67\cite{Quattoni:2009:RIS} without fine-tune on this dataset.
For comparison, we implement using only 4096-dimensional feature vectors
extracted from Caffe without region proposals, spatial pyramid matching 
and max-pooling which has the mAP of 59.9507\%.

% overall results
We compare our scene classification tasks with the performance without perform
L2 normalization and the performance of using only the features extracted from
CNN, summarized in Table~\ref{tab:overall}. Around 14\% improvement of using
region proposals, max-pooling and spatial pyramid are shown. In majority of
categories, we perform much better on the average precision. Some examples
are shoes shop, bedroom, bowling, grocery store, hospital room and operating
room. This might due to the region proposals and spatial pyramid technique
allow us to better characterize the particular objects belong to the category.
However, there are also some drops of average accuracy using our methods and
mainly for these three categories: prison-cell, library and living room. These
three categories are all relatively easier to be characterized by global
spatial properties.

\iffalse
% Table 2
\begin{table}[ht]
        \caption{Average Acurracy that Drops}
        \centering
        \begin{tabular}{l c c}
        \hline \hline
        Category    & our method & only feature \\ \hline
        prison-cell & 45\%       & 70\% \\
        library     & 45\%       & 70\% \\
        livingroom  & 20\%       & 40\% \\
        \hline
        \end{tabular}
        \label{tab:overall}
\end{table}
\fi
% compare with other paper results

% heatmap visulizations
