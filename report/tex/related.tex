%provide a detailed description of related papers (not necessarily limited to
%those in the schedule).  If you're proposing a new idea or extending an existing
%approach, compare and contrast it with existing work.  If you're analyzing one
%or two related techniques, describe how they relate to other relevant work;

Scene classification means to provide information about the semantic category
or the function of a given image. Among different kind of scene classification
tasks, the indoor scene classification is considered to be one of the most
difficult since the lack of discriminative features and contexts at the high
level~\cite{Quattoni:2009:RIS}. Spatial pyramid
representation\cite{Lazebnik:2006:BBF} is a popular method used for scene
classification tasks. It is a simple and computationally efficient extension of
an orderless bag-of-features image representation. However, without a proper
high-level feature representation, such schemes often fail to offer sufficient
semantic information of a scene. Object bank\cite{Li:2010:OBA} is among the
first to propose a high-level image representation for scene classification. It
uses a large number of pre-trained generic object detectors to create response
maps for high level visual recognition tasks. The combination of off-the-shelf
object detectors and a simple linear prediction model with a sparse-coding
scheme achieves superior predictive power over similar linear prediction models
trained on conventional representations. However, this method also limits the
performance of their system to the performance of the object detectors they
choose. Recently, Convolutional Neural Networks (CNNs) with flexible capacity
makes training from large-scale dataset such as ImageNet~\cite{Deng:2009:IAL}
possible. In the work of A. Krizhevsky et al.\cite{Krizhevsky:2012:ICD}, they
trained one of the largest CNNs on the subsets of ImageNet and achieved better
results than any other state-of-the-art methods in 2012. While their CNN system
focuses on object detection, the features generated can be used for other
applications such as scene classification. Two types of improvements has been
done on top of their CNN works. The first type of improvement tries to address
the problem of generating possible object locations in an image. Selective
search method~\cite{Uijlings:2013:SSO} combines the strength of both an
exhaustive search and segmentation and results in a small set of data-driven,
class-independent, high quality locations. Girshick et al. propose the Regions
with CNN features (R-CNN) method~\cite{Girshick:2013:RFH} as a more effective
feature generation method. Alternatively, Zhou et al. try to increase the
performance of scene classification using CNN by creating a new scene-centric
database~\cite{Zhou:2014:LDF}.
